\documentclass{beamer}

\usetheme{Padova}

\title{WNMA Project}
\subtitle{Real-time crowd information using Bluetooth: a full-stack solution}
\author{Luca Marchiori}
\date{25 Marzo 2025}


\begin{document}

\maketitle

\begin{frame}{Outline}
	\tableofcontents
\end{frame}


\section{Introduction}

\begin{frame}{Introduction}
	\textbf{Project Idea:} is it possible to exploit Bluetooth to count how many people are there in a room / building and the occupancy trends?
	\begin{itemize}
		\item Seat availability in libraries (without reservation)\vspace{.5em}
		\item Workforce management (effective deployment)\vspace{.5em}
		\item Health-critical monitoring (pandemic)\vspace{.5em}
	\end{itemize}

	\textbf{Assumption:} BT is a very diffused technology and nowadays most people have a BT-enabled device (smartphone, smartwatch, etc.) with them. Often it is turned on beacause of low energy consumption.

\end{frame}

\section{Technology stack}

\begin{frame}{Scanner}
	The scanner is a device that periodically scans \footnotemark the environment for Bluetooth devices and sends the data to the server.
	\vspace{1em}
	Implemented in Go, can run both on Raspberry Pi and Arduino\footnotemark.
	\begin{block}
		{Features}
		\begin{itemize}
			\item Low energy consumption
			\item Low cost hardware
			\item Easy deployment
		\end{itemize}
	\end{block}

	Thanks to linux's crontab, the scanner can be scheduled to run at specific times, e.g. every 5 minutes.

	\footnotetext[1]{Use the go-bluetooth library and the Bluez DBus API}
	\footnotetext[2]{Can be compiled for Arduino using TinyGo}
\end{frame}
\begin{frame}{Server}
	The server includes both a backend and a frontend developed in a product-ready fashion.
	\begin{block}
		{Backend}
		\begin{itemize}
			\item Implemented in Go
			\item RESTful API
			\item Data storage: SQLite
		\end{itemize}
	\end{block}

	\begin{block}
		{Frontend}
		\begin{itemize}
			\item Implemented in React
			\item Real-time data visualization
		\end{itemize}
	\end{block}

\end{frame}

\section{System Architecture}

\begin{frame}{System Architecture}
	\begin{figure}
		\centering
		\includegraphics[width=0.7\textwidth]{images/WNMA-ProjectScheme.jpg}
		\caption{System architecture}
	\end{figure}
\end{frame}

\section{Field test}
\begin{frame}{Field test}
	The system has been tested in a real environment: a small local library.
	\begin{itemize}
		\item The scanner (Raspberry Pi) has been placed in a central position
		\item To avoid hosting costs, the server has been deployed on the Raspberry loopback interface
		\item Three days of data collection with few people in the library
	\end{itemize}
\end{frame}

\begin{frame}{Field test}
	\begin{columns}
		% Column 1
		\begin{column}{0.5\textwidth}
			\begin{figure}
				\centering
				\includegraphics[width=1\textwidth]{images/RpiBiblioPhoto.jpg}
			\end{figure}
		\end{column}
		% Column 2    
		\begin{column}{0.5\textwidth}
			\begin{figure}
				\centering
				\includegraphics[width=1\textwidth]{images/ScanningScreen.png}
			\end{figure}
		\end{column}
	\end{columns}


\end{frame}

\end{document}
